\documentclass{article}

\usepackage{csc505}

\begin{document}

\HD{Homework \#2}

\begin{center}
  \textbf{due Sunday, 24 February 2019, at 11:00 PM}
\end{center}

\emph{Note: If you choose to work with a partner on this homework, you need
  to register by 5:00 PM on January 18.
  Instructions for doing so will be given later, but you will need to follow
  them exactly. You are allowed to work with a partner in a different section.}

\emph{When you submit your assignment solutions please take the time to
  indicate the starting page of each answer. To make life easy for the TA's
  put the answer to each problem on a separate page (you may put multiple
  parts of the same problem on one page)}

\emph{If you are working with a partner, please submit \textbf{one solution
    only}. GradeScope allows you to specify your partner after a
  submission. Moodle does not, but we will grade \textbf{one solution per
    team only.}}

Submissions for the written part of the homework, including your
analysis of the programs should be submitted to GradeScope (these will be
set up as two distinct assignments -- \texttt{Homework~2} and
\texttt{Program~2~Analysis}). The source
files for your programs should be submitted to \texttt{Assignment~2} in Moodle.

\begin{enumerate}
\item \cmt{9 points: 3 for (a), 4 for (b), and 2 for (c)}
  \emph{Purpose: Understanding Heapsort and lower bounds on algorithms.}

  Consider the special case where Heapsort is used to sort 0's and~1's.
  Because Heapsort is comparison-based, you should assume that Heapsort
  does not take any special advantage of the values of the keys.
  Let $k$ be the number of~1's in the array.
  All bounds that follow are expressed as functions of \emph{both} $n$,
  the total number of elements, and $k$, the number of~1's.
  Assume $k < n/2$. Recall that the MakeHeap phase takes linear time, so
  no need to say anything about it in your proof.

  \begin{enumerate}
  \item Prove that the worst case number of key comparisons in the
    situation described above is \OH{n + k \lg n}.
    This is linear unless $k \in \omega(n/\lg n)$.

  \item
    Show that the above bound is 'tight' in the sense that it is also a
    $\Omega$ bound for the algorithm.  Recall that to prove a worst case
    lower bound of \LB{g(n)} for an algorithm we need to show that
    there exists $c,n_0 > 0$, so that for each $n \geq n_0$,
    there exists an input of size $n$ that causes the algorithm to take
    time (or, in this case, number of comparisons), $\geq c g(n)$.
    Where a time bound has two parameters, the argument needs to work
    for any valid combination of the two.

  \item
    Prove that the number of key comparisons depends on the original position of the
    1's in the array? This is not about the worst case or an
    asymptotic bound. The question refers to how the exact number of comparisons
    depends on the positions of the~1's.
    Since your proof will require an example with duplicate keys, you should use
    \textsf{key,value} pairs\footnote{See the programming assignment
      below.}
    to distinguish between elements. Make your example as small as possible.

  \end{enumerate}

  \newpage
\item \cmt{10 points: 2 for (a), 3 for (b), and 5 for (c)} \emph{Purpose: understanding sorting algorithms and the sorting lower bound.}
  Problem 8-4 on pages 206--207 (179--180 in 2/e): red and blue jugs.

  \newpage
\item \cmt{6 points, 3 points each part}
  \emph{Understanding the analysis of linear time selection.}
  Exercise~\textbf{9.3-1} on page~223 (page~192 in 2/e): selection with
  groups of size other than five.
  
  
\end{enumerate} % end, problems

\label{last}
\end{document}

% [Last modified: 2019 02 12 at 13:17:04 GMT]
